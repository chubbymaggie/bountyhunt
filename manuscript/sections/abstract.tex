Bug bounty programs offer a modern platform for organizations to crowdsource their Internet security, for security researchers to be fairly rewarded for their contributions, and often, for the public disclosure of the ``true" value of vulnerabilities. Little is known however regarding the incentives set by bug bounty programs drive new bug submissions, and thus to overall security improvement through the progressive exhaustion of findable vulnerabilities. Here, we recognize the importance of {\it timed incentives}, which drive opportunities and behaviors for security researchers. Each new bug bounty program, launched by a specific organization, creates an new set of opportunities, characterized by a pool of {\it findable} vulnerabilities, which drive the progressive -- yet increasingly hard -- discovery process. Using public data from a leading bug bounty platform, we find that the set of findable vulnerabilities generally (i.e., all programs taken together) exhausts following a power law decay $\sim t^{-\alpha}$ with $\alpha \approx 0.29(3)$, while the average amount of reward increases linearly, starting from \$278 and adding nearly \$2 of reward at each new successful submission. We question however the importance of this increased incentive for security researchers facing discovery of increasingly less obvious vulnerabilities, and left with two possible strategies: further harvesting in the same bug bounty program (with increased difficulty) or turning to a new fresh program with more obvious bugs. Our results show that while the chance of finding a security researcher having found more than $x$ vulnerabilities in a specific program follows power law distribution given by $P(X>x) \sim 1/x^{\mu}$ with $\mu \approx 1.6$ (at the aggregate level, but with some variations across programs), the reward for a researcher finding an additional bug increases exponentially by approximately 3\%. The {\it exploration} alternative option depends on newly launched programs, which occur at a rate of approximately one program every two months. From these figures, we can compute the expected utility of security researcher when deciding whether to continue {\it harvesting} in the same program or to {\it explore} a freshly created program.